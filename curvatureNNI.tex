\documentclass{amsart}

\usepackage{amsmath,amsthm}
\usepackage{todonotes}
\usepackage[notref,notcite]{showkeys} % show keys for eqs, etc.
\usepackage{cite}
\usepackage{enumerate}
\usepackage[colorlinks=true]{hyperref}

\newtheorem{lemma}{Lemma}
\newtheorem{question}{Question}
\newtheorem{proposition}{Proposition}

\newcommand{\tN}{\mathrm{\tau NNI}}

\begin{document}

\section{Discrete $\tau$-space}
%%%%%%%%%%%%%%%%%%%%%%%%%%%%%%%%%%%%%%%%%

We consider the ranked-NNI graph (the distance on ranked topologies inherited from 
$\tau$-space). 

\begin{lemma}
Let $T$ be a ranked tree on $n>2$ leaves. Then \[n-1\leq \deg(T)\leq2(n-2).\] 
\end{lemma}

\proof
Obvious. 
\endproof

\begin{lemma}
\begin{enumerate}[(1)]
\item $deg(T)-deg(R) \leq n-3.$
\item $\dfrac{\deg(T)}{deg(R)}>\dfrac12.$
\item $\lim\limits_n\dfrac{\deg(T)}{deg(R)}=\dfrac{1}{2}.$
\end{enumerate}
\end{lemma}

\proof
Follows from previous Lemma. 
\endproof

\begin{lemma}
$d_{\tau}(T,R) = 1 \Rightarrow |N_1(T)\cap N_1(R)|\in\{0,1\}.$
\end{lemma}

\proof
Obvious.
\endproof

\begin{lemma}
Under random walk, the following is true for any finite metrics $d$ and any points $x,y$:
\[
\dfrac{-2}{d(x,y)} \leq \kappa(x,y) \leq \dfrac{2}{d(x,y)}.
\]
\end{lemma}

\proof
Obvious, assuming the measure of any point is bounded by one. 
\endproof


\begin{lemma}
Assume uniform random walk and $d_\tau(T,R) = 1$. Then \[\kappa(T,R) \leq 
\dfrac{1}{2(n-2)}.\]
\end{lemma}

\proof
Follows from Lemma~5.2 in~\cite{WhiddenMatsen}. 
\endproof

What does this Lemma say about full and topological $\tau$-spaces? In full $\tau$-space,
$\kappa(T,R) \leq 0$~\cite{GD} for some random walk. For which one? 

\begin{proposition}
Let $(\mathcal G_n)_{n \in \omega}$ be a sequence of finite graphs and 
$(x_n, y_n)$ a sequence of adjacent vertices from $\mathcal G_n$ such that
\begin{enumerate}[(1)]
\item $\big|N(x_n) \cap N(y_n)\big| = o(|N(x_n)|).$ 
\item $\big||N(x_n)| - |N(y_n)|\big| = o(|N(x_n)|).$ 
\item $\big|\{(x,y) \mid x \in N(x_n),~ y \in N(y_n),~ d(x, y) \geq 2\}\big| = o(|N(x_n)|).$
\end{enumerate}

Then $\lim\limits_{n \to \infty} \kappa(x_n, y_n) = 0.$
\end{proposition}

\proof

\endproof

\begin{question}
Assume some (e.g.\ uniform, lazy, ...) random walk. What are the $m$ and $M$ 
such that 
\[
m(\deg x,\deg y, \ldots) \leq \kappa(x,y) \leq M(\deg x, \deg y, \ldots)? 
\]
\end{question}

\section{Full $\tau$-space}
%%%%%%%%%%%%%%%%%%%%%%%%%%%%%%%%%%%%%%%%%

We now extend the space of possible moves by including branch length moves. In general,
such moves should allow the branch length to be any positive real number, but this setting
seems to require quite sophisticated techniques for computing the curvature and hence is 
a direction of future research. Instead, here we allow branch length (or rather $\tau$-
coordinates) to be of two possible values: long (denoted by $\Lambda$) and short (denoted 
by $\lambda$). Node that the number of possible `branch lengths' can be generalised 
to any finite set $\lambda_1,\ldots,\lambda_s$\todo{It's worth pointing out explicitly what
the change in what result will be.}.

More precisely, we consider a graph $\tN$ on the set of ranked rooted tree topologies 
(which we call simply trees from now on) with
all leaves of rank $0$ and all intervals between nodes of the tree marked by $\Lambda$ or
$\lambda$. Two trees are adjacent in $\tN$ if and only if they are in adjacent orthants in 
$\tau$-space~\cite{GD} and the interval corresponding to the boundary in $\tau$-space
is marked by $\lambda$ in both trees, or they are of the same ranked topology and there 
exists precisely one interval which is marked by different symbols in the two trees. 

The intuition behind this notion is that one move corresponds to either 
a $\tau$-move from a short interval to a new short interval or a change of the length of 
precisely one interval. 

First, we consider a uniform (lazy) random walk, where given a tree $T$ we do nothing 
with probability $p$, where $0\leq p\leq 1$ is the probability of \href{https://academichelp.net/wp-content/uploads/2014/01/laziness.jpg}{leziness}, and uniformly move to a neighbouring tree with probability $1-p$. 

\begin{lemma}
Let $T$ be a ranked tree on $n>2$ leaves. Then \[n-2\leq \deg(T)\leq3(n-2).\] 
\end{lemma}

\proof
Obvious.
\endproof

\begin{lemma}
\begin{enumerate}[(1)]
\item $deg(T)-deg(R) \leq 2(n-2).$
\item $\min\limits_{T,R}\dfrac{\deg(T)}{deg(R)} = \dfrac13$ for all $n.$
\end{enumerate}
\end{lemma}

\proof
Follows from previous Lemma. 
\endproof

These two lemmas remains unchanged if we change the number of branch lengths. 

Other lemmas follow similarly too. 

\end{document}
